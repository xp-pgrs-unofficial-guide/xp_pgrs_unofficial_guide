%# -*- coding: utf-8-unix -*-
%%==================================================

\chapter{其他攻略}

\begin{flushright}
    ——看一眼不亏,看两眼血赚
\end{flushright}

\section{大坑:假如明天你突然丢失电脑所有数据怎么办?——如何备份资料}

\begin{figure}[H]
    \begin{tabular}{rl}
        \includegraphics[width=0.5\columnwidth]{author-folder/Kai.Wu/backup1.jpg} &
        \includegraphics[width=0.4\columnwidth]{author-folder/Kai.Wu/backup2.jpg}
    \end{tabular}
    \caption{\href{https://baijiahao.baidu.com/s?id=1719578217211021768}{链接:贵阳一女大学生8000字毕业论文保存失败崩溃大哭}}
\end{figure}

绝大部分同学本科就买了自己的电脑。虽然自己的各种作业资料,到现在博士阶段的研究数据、论文,全部就只存放在这一台电脑里,但并没有足够意识到自己电脑的重要,和保存在电脑里信息的脆弱。每年,都会有如上图的事情发生,给当事人带来上新闻被全国人民笑话的沉重打击。

丢失一篇正在写的论文,其实是小事,毕竟研究资料全部都在。上面新闻的当事人,其实也就用了几天时间就重新写好了论文。但那些万年不动、又极其重要的资料,又是绝对安全的吗?

想象一下,假如
\begin{itemize}
    \item 你某天手一滑,电脑摔倒地上,再也开不了机,硬盘也没法恢复
    \item 头昏脑涨的一天,你接了一杯咖啡,一不注意,咖啡撒在了电脑上,瞬间短路,电脑滋滋冒油
    \item 你不慎点击了一个链接,中了勒索病毒;或者,西浦内网被网络攻击攻陷,所有电脑中勒索病毒。你的所有文件被加密封锁。
    \item 你把电脑带出去办公或者采集数据,整个电脑直接被偷走
    \item 你过年回家,在路上,背包或者行李箱被人偷走了或者拿错了
    \item 无恶意,但万一,西浦着火了,西浦进水了(MB的5楼一下大雨就漏水),西浦楼塌了,撤离的时候你根本来不及拿电脑
\end{itemize}


\begin{figure}[H]
    \centering
    \includegraphics[width=0.8\columnwidth]{author-folder/Kai.Wu/backup_zhihu.jpg}
    \caption{\url{https://www.zhihu.com/question/394796969/answer/1501840215}}
\end{figure}

这些情况,单个可能很罕见,但所有可能性加起来,再乘以3到4年时间,至少发生一次的概率并不小。如果发生,你的博士进度会被耽误多久,你的青春又会被耽误多久呢。

只有一份的数据是非常脆弱的。现在思考一下,你赖以生存的电脑、数据、论文,是否只存了一份?如果是,请立即把屁股从椅子上挪起来,去厕所洗把脸,问问自己都是成年人了,怎么心这么大,然后开始看下面的攻略。

\subsection{理论:321原则}

「3-2-1 原则」是商业公司通用的数据备份金方法,具体内容是:
\begin{itemize}
    \item 3:存储 3 份完整文件,一份原件加上两份拷贝。
    \item 2:将文件起码保持在两种不同的介质上。
    \item 1:将一份拷贝保存在异地。
\end{itemize}

所谓的「介质」,是指内置硬盘、移动硬盘、U盘、网盘等不同的存储介质。说起来好像很麻烦,但对我们来说最简单的实践是:原件保存在内置硬盘,将两份拷贝分别存到移动硬盘和网盘,就同时达成了三条要求。这时,你回过头去看一下我上面列举的所有可能的灾难性问题(甚至即使是网盘公司突然倒闭),都能做到游刃有余,不影响你的科研!

\subsection{实践}

到动手准备的环节了。首先根据你手里的资源,决定你要备份的范围。最土豪的情况,你可以把整个电脑做一个本地备份 + 云端网盘备份,但很多时候,要么网盘容量很小,要么移动硬盘容量不够,要么网速太慢把太多资料传网上很慢,所以先划定你的备份范围。

\begin{itemize}
    \item 最经济的方案:只备份最重要的文件。例如:论文、科研资料、CV、电脑桌面,等等。网盘实时备份+移动硬盘定期备份。对移动硬盘和网盘容量要求都很小。
    \item 最高效的方案:整个电脑(包括最重要的数据)本地备份;最重要的科研、论文等加一分网盘实时备份。这需要一块比你电脑硬盘更大的移动硬盘(不贵),加正常容量(10个G左右)的网盘。
    \item \sout{最土豪的方案:买个NAS,插几十T的盘,组RAID1,再实时传一份到异地。}
\end{itemize}

正常电脑硬盘容量一般在2T内,另外我假设大家去买一块2T以上的移动硬盘(500块以内)应该压力不大(导师经费较多的,可以直接问导师要,直接说想要一块移动硬盘做备份问导师能不能帮忙买就行,2000块以内报销都很简单)。所以下面只介绍中间的高效的方法

\subsubsection{电脑整机备份}
\label{sec.pc_backup}

把电脑整机备份的一大好处就是,假设电脑直接灭失,新买一台电脑过后,能在一天之内完全恢复之前的工作状态,无缝衔接,科研毫不受影响。

Windows和macOS都自带系统级的整机备份方案。网上攻略特别多,我就不废话了。
\begin{itemize}
    \item Win请在b站直接搜索windows备份看教程,一般使用系统的“备份与还原(windows7)”(虽然名字里有win7但win10和win11也是完全能用的,这也就是为什么一直没被扔掉),比如看这个视频 \url{https://www.bilibili.com/video/BV1Dy4y1x7RP}
    \item macOS请在b站直接搜索mac time machine,一般使用系统设置里的“Time Machine”,比如看这个视频 \url{https://www.bilibili.com/video/BV1oy4y177qS}
\end{itemize}
一个视频看了不太明白没关系,多搜几个看,也同时善用谷歌、知乎等。初学者可能感觉学习成本有点高,但不要等到电脑炸了再开始准备呀。

\subsubsection{网盘备份}
\label{sec.net_drive_backup}

Box(\url{https://box.xjtlu.edu.cn/})是学校的自建网盘,2022年升级到了每人100G的免费空间,内网上传下载速度可达千兆,同时自带3个月的文件历史版本,秒杀一众网盘。把重要的文件夹拖入seafile客户端(不用移动文件夹),他就会自动实时备份到学校网盘。使用方法见 \url{https://guide.xjtlu.edu.cn/box/student/drive_client/drive_clent_for_windows.html} 。Box唯一的问题是不支持链接分享文件(老师有分享权限,我们没有),但作为备份网盘非常合适。

另外,如果对学校IT不放心,可以同时安装一个另外的商业网盘进行第二重网盘备份。个人严重不推荐百度云,因为偶尔会随机吃文件。国内的,我个人推荐坚果云(免费版无限空间,但限制每月流量),国外网盘推荐Onedrive(个人版免费5G,可在淘宝搜索onedrive扩容花几块钱扩容到永久15G)。另外,大家的利物浦账号下都有免费1T的Onedrive,羊毛薅起来(但注意毕业前得把数据迁出去,因为毕业了要收回账号),参见这一章节 \ref{sec.fuli_liverpool}。


\subsection{论文备份·文件历史版本}
\begin{newminipage}[0.65]
    上述备份防灾的方法可以应对绝大多数数据丢失的场景。但,各位的论文,包括毕业论文和期刊发表的论文,作为最重要也是最脆弱的资料,除了怕丢失还特别怕这两件灭顶之灾:
    \begin{enumerate}
        \item 版本混乱。论文从一开始写到写完,中间修改个几十版是常事。特别是在临近提交的时候,一定会由于导师意见、二导意见、合作者意见等再battle个十几版出来。其中各版本的区别可能非常小,肉眼也很难分辨。一定会有一天,你也搞不清楚哪个是哪个版本、忘了在哪个版本作了什么修改,会在分辨版本上反复花费时间,进而导致 ↓
        \item 文件覆盖,例如(1)新版本直接覆盖了老版本,但过一段时间又因为某种原因要拿回老版本 (2)老版本覆盖了新版本,几个月的论文白改
    \end{enumerate}
\end{newminipage}
\begin{newminipage}[0.34]
    \begin{figure}[H]
        % \caption{导师:把最终版发给我}
        \includegraphics[width=0.95\columnwidth, right]{author-folder/Kai.Wu/thesis_versions.jpg}
    \end{figure}
\end{newminipage}

读到这里,你如果感兴趣可以做个实验,创建一个文件,里面写上一些内容。然后在其他文件夹新建一个同名文件,写上另一些内容。最后,把它拖过来,覆盖掉前面的文件,这时,试一试自己有没有任何办法找回前面文件的内容。

你可能要问,上面不是说了这么多备份方法吗?是的,这些备份方法主要的作用是防丢失,比如电脑丢失、硬盘损坏,但少有备份能做到防覆盖。比如,实时备份的网盘,你本地把文件错误覆盖了,云端也马上被覆盖了。本地备份硬盘,如果备份频率不高,则还有一线生机,但如果过一段时间再去找,也可能找不到了。

因此,为了你的论文(命根子)安全,完全值得再加一层保险:

\begin{enumerate}
    \item Windows开启系统级的文件历史保护:\url{https://www.asus.com.cn/support/FAQ/1013067/},别忘了一定要把你的论文文件夹添加进去才能生效
    \item mac的TimeMachine(时间机器)备份自带了文件历史功能。参照前面的 \ref{sec.pc_backup} 配置好TM,整个电脑的文件历史就都包括在备份里了。
    \item Linux用户,我相信你们有的是办法。最笨的办法,直接写一个shell脚本,把论文文件夹直接复制到其他目录、用日期命名,最后把这个脚本加到crontab里定期运行即可。
    \item 以上是开启本地文件历史的方法。一些网盘自带文件版本功能,但一般限制版本数量或者日期,或者必须要买会员才能开启文件版本。开启过后,相当于云端也存了一份版本历史,万一本地的你没设置好或者其他什么原因不能用,还能在错误覆盖的时候救你一命。学校的seadrive自带大概两个月的文件版本,免费,非常良心,再次建议大家使用。直接把文件夹拖进去备份,就会自动产生文件版本。参见章节 \ref{sec.net_drive_backup} 。
    \item 会一点Git、GitHub的同学,请直接把整个论文文件夹创建成Git项目,并上传到GitHub做成private repo,每天工作结束后交一个commit来备份版本,同时记录你修改了什么内容。对于论文,Git是最完美的备份方案,可以轻松回滚到老版本同时从不担心覆盖、甚至能轻松知道哪个地方的修改是在什么时候引入的,比以上所有方法都靠谱。Overleaf也有GitHub集成,可以方便的与导师协作。但Git学习成本略高,不过网上有无数详尽的教程,感兴趣、学有余力的同学可自行学习。
\end{enumerate}


\begin{flushright}
(2022年12月02日 by \Wu)

(major update: 2022年12月30日 by \Wu)
\end{flushright}

% \begin{figure}[H]
%     \centering
%     \includegraphics[width=0.5\columnwidth]{author-folder/Kai.Wu/}
% \end{figure}


% \usepackage[export]{adjustbox}

% \item 
% \begin{minipage}{0.3\textwidth}
%     文字
% \end{minipage}
% \begin{minipage}{0.63\textwidth}
%     \begin{figure}[H]
%         \includegraphics[width=0.95\columnwidth, right]{author-folder/Kai.Wu/}
%     \end{figure}
% \end{minipage}

% \input{author-folder/Kai.Wu/.tex}
 \clearpage

\section{远程控制·如何校外访问校内电脑}

人在校外的时候,经常会很想念校内的电脑。不管是学校给大家每人配的小台式,还是用导师经费买的的其他电脑,都可以远程控制。

\subsection{学校配的电脑·准备工作}
(如果不是想访问学校配的电脑请直接跳到下一节)

学校给的电脑比较特殊,默认我们不能安装软件。有两种办法
\begin{enumerate}
    \item 发邮件或者打电话给IT,直接问他们要管理员权限。就说你是博士生,要在学校给的电脑上装软件。之后就可以装控制软件了
    \item 上面这种办法过后,电脑还是归学校管理,仍然有一些限制,但对普通使用基本够了。\sout{(有强迫症)}想完全控制学校电脑的同学\sout{(比如我)},可以直接格盘重装系统,或者分区(保留原系统)另装系统做双系统。windows重装系统、分区做双系统的教程,b站和知乎特别多。
\end{enumerate}

\subsection{桌面控制软件推荐}
以下软件不分系统,win/linux/mac均可用。
\begin{itemize}
    \item VNC: 被控端(学校电脑)安装 VNC Server \url{https://www.realvnc.com/en/connect/download/vnc/},控制端(你手上的电脑)安装 VNC Viewer \url{https://www.realvnc.com/en/connect/download/viewer/},注册账号登录使用,无需记录连接码。推荐理由:VNC是饱经检验的远程桌面,只要网络不挂,一般不会有问题。
    \item ToDesk:\url{https://www.todesk.com/}两边电脑软件一样。推荐理由:新兴软件,流畅度一般比vnc好一些。
    \item 其他:anydesk,Teamviewer,向日葵,均可尝试。其中,不太推荐Teamviewer,虽然老牌,但因为很可能检测到西浦的网络环境过后强制购买商业版,其他软件无此问题。
\end{itemize}

\subsection{远程SSH}
(不用linux的同学可直接跳到下一节看踩坑)

如果被控端是linux,你又主要使用命令行操作,那在桌面控制外,直接设法连接SSH,速度会快特别多。

电脑在学校是10开头的内网ip,如何在外面SSH?这种奇技淫巧叫做【内网穿透】

首先推荐这个视频:

\href{https://www.bilibili.com/video/BV1Qq4y1w7F5}{【硬核】公网访问?内网穿透!零经验上手!}\url{https://www.bilibili.com/video/BV1Qq4y1w7F5}

视频里校内机器可用的方案有两个,分别是

\begin{itemize}
    \item 视频的04:52,用IPV6连接。但要注意,我实测学校IPV6不稳定,过几天就会出现有v6地址但不能上v6网站的情况,更别提访问。即使你能做ddns,我个人也不推荐。
    \item 更靠谱的是视频08:09介绍的大内网穿透。免费易用的方案有:Zerotier(在境外,略慢),花生壳(国内老牌,但注册需要身份证),NOFRP(新兴,可靠性可能不高),或者直接在b站搜【免费内网穿透】,会有很多新方案。这几个方案如果不太会,b站、知乎有大量教程。
    \item (进阶大流量版,但学习成本较高)免费的方案对流量和带宽的限制都比较大,只是SSH执行命令完全足够了。但如果你想要奢侈的用scp拷文件,或者用SSH转发VNC,请考虑用导师的钱租一台云服务器,手动搭建frp服务(比较麻烦,但操作完过后用起来很爽,转发VNC比前面几种远程桌面都快),甚至,在宿舍宽带下弄个垃圾机器做跳板机,搞到公网ip,配合ddns,无限流量访问。这些方案学习成本较高,有兴趣的可以参考网上教程慢慢折腾。
\end{itemize}

\subsection{踩过的坑}
下面是我踩过大量坑过后从坑里带出来的经验:
\begin{enumerate}
    \item 冗余提高可靠性:不要只用一个远程控制方案,万一挂了一个,还能用另一个。请至少安装两个,如果是长时间不在校,可以安装更多个。我自己用的是:家宽里用一台二手瘦主机搭frp加ddns + 花生壳作backup-plan + vnc作another-backup-plan + ToDesk四重备份方案,稳定访问校内的linux电脑。
    \item 安装完成后,重启一次,看下控制软件是否能自己启动。
    \item 如果被控端是windows电脑,请务必屏蔽系统更新。虽然重启没问题,但windows大版本更新过后,会卡在一个让你同意新的用户协议的界面,除非你到场点一下,所有软件都不会启动,也就直接失去控制。请百度【禁止windows更新】。个人建议用组策略+改host组合拳。
    \item 跑程序的同学,请务必注意,不要超内存,不要炸系统内存。内存一吃满,系统立即死机,控制软件全杀完,而且还不会自己重启,必须要到场强制重启。请一定想办法在程序里监测、控制内存用量。
    \item 有条件的可以购买【向日葵控控】这样的远程KVM硬件,即使死机也可以自己远程强制重启。
    \item 再好的方案,也敌不过学校网络维护断网和学校施工断电。而这些,也是可以防御的。应对断网,请注意在网络登录时勾选macauth,网络维护完了就又能用了。来电自动启动功能,需要主板支持,在主板说明书搜索boot on power或boot with power,或直接搜power,看看有没有。或者,购买向日葵控控。不过断电断网始终是极少数时候,除非你一走几个月,基本不需要考虑\sout{(但万一呢)}。
    \item 请和至少一个在校同学搞好关系,新手总会遇到各种意外的情况失去控制,需要办公室同学帮你跑腿看机器情况。100种远程控制的方案背后,也需要好同学做物理控制。万一的万一,哪天网线被人不小心踢掉了呢,还是只有\sout{通过py交易}让同学帮你弄。
\end{enumerate}

\begin{flushright}
(2022年11月09日 by \Wu)
\end{flushright}

% \begin{figure}[H]
%     \centering
%     \includegraphics[width=0.5\columnwidth]{author-folder/Kai.Wu/}
% \end{figure}


% \usepackage[export]{adjustbox}

% \item 
% \begin{minipage}{0.3\textwidth}
%     文字
% \end{minipage}
% \begin{minipage}{0.63\textwidth}
%     \begin{figure}[H]
%         \includegraphics[width=0.95\columnwidth, right]{author-folder/Kai.Wu/}
%     \end{figure}
% \end{minipage}

% \input{author-folder/Kai.Wu/.tex}
 \clearpage

\section{我的老腰受不了了——如何保持颈椎、腰椎健康}

\subsection{缓解腰的问题:升降桌}

\begin{newminipage}[0.39]
    \begin{figure}[H]
        \includegraphics[width=0.95\columnwidth, right]{author-folder/Yue.Zhou/gongzuotai.jpg}
    \end{figure}
\end{newminipage}
\begin{newminipage}[0.6]
    在没有读博之前,我就患上了严重的腰肌劳损。这个玩意怎么说呢,说是大病吧,他不是,但无时无刻不在折磨着你。最严重的时候,疼的我晚上都睡不着。去医院看过也开过药,但医生说这个病需要自己保养,无法根治。后来我试着跑了半年步,哎!腰肌劳损神奇的好了,现在一点都不疼了(如遇久坐,仍旧是不行的,会疼)。读博之后也没有心情运动了,于是买了一个升降工作台(推荐人手一个!太好用了!)。可以站着办公,站累了就坐下办公。极大程度上杜绝了久坐对腰椎的伤害。
\end{newminipage}

\begin{flushright}
(2022年12月30日 by \Yue)
\end{flushright}

\subsection{来自重度腰椎间盘突出、颈椎病患者的深度建议}

\subsubsection{升降桌:可极大缓解腰椎问题,但不能完全解决}

我十年腰椎间盘突出加五年颈椎病,19年的时候腰就疼到没法坐一整天,就在实验室放了台升降桌,站着用了三年电脑。今年还在卧室里放了同款升降桌。但长期站着和坐着一样对腰不好,最后会变成站着和坐着一样会腰疼。升降桌不是长久之计,站着只是换了角度继续损耗腰椎。

不能完全依赖升降桌:这几年太依赖升降桌,导致现在长时间站着和坐着一样腰疼,去拍核磁医生都问我是不是腰骨折了。如果真的想缓解腰疼,就要避免久坐久站。中学教师容易得腰疼一方面就是因为站着讲课时间长了。

总之升降桌可以用,但不能长期用它代替坐着用电脑。可以交替用,平时再多锻炼就行。

注意事项:升降桌记得没事调一调高度。我放实验室里的升降桌几年没调好像液压杆生锈了,都调不动了,直接变成了一般桌子。

\subsubsection{外接屏:避免和缓解颈椎病}

颈椎病难受的一点就是,低头时间长就脖子疼,出门在地铁高铁上我都不得不把手机竖起来抬高和眼平行,不然用不了太长时间。

我再贡献一个避免和缓解颈椎病的方法,就是不要长时间用笔记本电脑,长时间低头看屏幕会演变成颈椎病。低头看手机也是。

如果常用笔记本电脑就要在卧室和办公室都放一个台式机屏幕当外接屏。

我颈椎病就是高强度用了一年笔记本电脑得上的,后来只用外接屏就不再继续加重了。

如果经常高强度用笔记本电脑,不用外接屏的话和整天低头用手机一样都是颈椎杀手。我在实验室看见有人长时间用笔记本都会劝他们用外接屏。

笔记本支架也可以,比显示器便宜。但如果视力不好,还是尽量外接屏,因为笔记本屏幕太小了,费眼。我以前也买了很多笔记本支架,现在都用来放显示器了,笔记本刚好能放在下面很方便。

\begin{figure}[H]
    \caption{我自己在卧室里就是升降桌,配笔记本支架。这个组合很适合我这种主要用笔记本的}
    \includegraphics[width=0.95\columnwidth, center]{author-folder/Jialin.Wang/shengjiangzhuo.jpeg}
\end{figure}


% \subsubsection{进一步缓解:趴着用电脑}

% 我今年起站也不能站太长时间,已经开始探索平躺和俯卧用电脑了。平躺我试过几天就放弃了,容易睡着。我觉得平躺就适合看看电影玩游戏。

% 俯卧我最近试了一个月,感觉还行,起码支起上半身不会犯困。

% 我是放了一个便携屏幕在床边,然后配一个趴枕,这样能趴一整天。趴枕好处是吃完饭趴着不会压迫腹部导致不适。

% \begin{figure}[H]
%     % \centering
%     \includegraphics[width=0.3\columnwidth, center]{author-folder/Jialin.Wang/pazhen.jpeg}
% \end{figure}

% 然后配一个可以调角度的笔记本支架放屏幕键盘和鼠标——其实直接放笔记本也行,但发热量大,搬来搬去不方便。用便携屏能快速在升降桌和俯卧之间切换——我也想过买一般的显示器支架,但没有卖那种适合在床上用的,而且一般显示器在床上用太大了。便携屏就是笔记本二手屏幕改装的,大小和笔记本电脑一样在床上用正好。

% 如果长时间趴着用电脑,我发现最好是用双臂稍微用力撑起上半身。但上半身主要还是靠着躯干受力,双臂是辅助作用,避免下巴受力太大。

% 我现在已经练成了能俯卧一天的神功,笔记本支架只要调好角度,让手和手臂平行,这样手腕就不会因为长时间弯曲而酸痛。

% 不过最近因为趴习惯了加上天冷就完全不锻炼,如果不趴着,还是腰疼。还是得坚持锻炼才行。

% 这一套方案我觉得真的完美适合我这种腰疼到无法长时间坐着和站着的人,但这些都不能治好腰疼。锻炼才是最重要的。时间长不锻炼,不常用的肌肉会萎缩,更不好。(我好想以后住在热带沿海地区,每天傍晚去海边游泳,傍晚的时候海水是温的)


\subsubsection{最好的解决方案:改变生活习惯、加强锻炼}

建议有腰椎间盘突出和颈椎病的人还是多锻炼。但不要盲目去健身房锻炼,最好的锻炼方法还是游泳,因为其他锻炼方式大多会需要腰部用力,加重腰疼。

我目前只是偶尔腰疼到不能弯腰。我还认识腰疼到走路一瘸一拐的,他也是慢慢锻炼才逐渐好转。

如果自制力不强,加学校游泳社团也行,起码有人经常一起去锻炼能有动力。

办卡的话,我记得每年九月份独墅湖游泳馆能办半价年卡,但学校社团很便宜,更划算。以前我办游泳卡前每周都跟着学校游泳社团去游一两次,办了年卡后,一年去了不到五次,后来就去得更少了。
如果自制力可以,每天锻炼是最好的方法。
如果坚持不下去,还无法改变生活习惯,就会越来越严重。

最后,如果腰疼比较严重,有空可以去医院骨科看看,做一下核磁,看一下腰椎到底发展到哪个地步,来决定怎么具体该恢复。现在腰疼在年轻人里很普及。

\begin{flushright}
(2022年12月30日 by Jialin Wang)
\end{flushright} \clearpage

\section{利物浦访学}
\label{section.UoL_visit}

\subsection{交流时长}
一般是3-6个月,短于3个月也行,原则上来说不允许超过6个月,如有超过六个月的,需要写明充足的理由,并获得两方导师以及 HoD 的批准。不过,利物浦那边学院的行政告诉我他们没有预算给超过六个月的部分,这个可能会是一个问题;另外,超过六个月的访学时间可能会涉及签证问题,需要提前咨询。

\subsection{关于 host supervisor}
如果导师团队中有利物浦导师,那默认就是那个老师。如果没有但又特别想去利物浦,可以自行联系利物浦那边的意向老师,要获得他/她的许可和签字。

尽管大多数同学的导师团队中会有一个来自利物浦的导师,但每个项目可能都不一样,也许你在去利物浦之前和他的互动并不频繁。这种不频繁可能是不同原因带来的,不一定是因为你工作的领域和他不同。所以如果你认为在利物浦与他合作会很有趣,那么请提前做些准备。

如果你之前和 ta 没什么互动,请务必通过某种方式建立一定程度上的熟悉感,这会使后面的合作更加顺畅。这可以是通过视频会议或邮件分享你的研究进展及计划、一起合作撰写 paper、或提前构思潜在的合作研究内容等。

我们知道团队合作有可能出于不同的原因(详见 \ref{subsection.teamwork}),但无论如何,请务必提前与你在利物浦的导师至少在大方向上确认你访问期间的活动,包括但不限于:需要他参与的事情、时间安排、需要你准备的事情、以及你准备做的事情。

\subsection{关于 funding}
西浦:学校提供1万2的经费,包含苏州往返上海办理签证的高铁票,签证费用,来回英国机票,英国当地往返利物浦的火车票(比如你落地曼城,坐火车去利物浦,就报销那一趟车票钱!以及回国时从利物浦前往机场的火车票or大巴车票),由于登机所必要的住宿。除了这些,其他的一概不报销!!而且总额不超过1.2w,超出的部分学校一概不负责。

利物浦:提供免费住宿(包bills,水电燃气随便用)+每周100磅的生活费


\subsection{签证流程}
\begin{itemize}
    \item 去官网填写相应信息
    \item 信息填写完毕后会让你选择递签地点和日期
    \item 会跳转到一个网站让你选择额外服务(比如加急或者快递),务必记住自己的申请号(GWF+一串数字)
    \item 准备相应材料,具体需要什么材料可以查看check list。所有材料均为英文。需加盖翻译公章或邀请你身边有专八证书的为你的翻译件签字 \\ \\
    护照是必备的,其他的材料包括但不限于:
    \begin{itemize}
        \item 银行流水/存款证明--我交的是存款证明,因为我在信息填写中写了存款金额,而且据说存款证明信息比较少,会加速审核流程。此处也可以递交固定资产证明。
        \item 结婚证(如果是已婚)
        \item 学校开的在读证明和资金支持(因为我是学生),已工作的人可以交工作证明
    \end{itemize}
    \item 递签--递签当日提前半小时到达签证中心即可。进去后会有工作人员引导你排队,然后进入里面将check list给工作人员,她会给你一个号码,等叫号。第一次叫号是初步询问,会问你要不要快递或短信或加急服务(短信服务可不要,因为邮箱第一时间也会有通知)。如果step 3未选择任何服务但这时要购买会给你开个单子然后让你去交钱。快递服务购买后会给你一个快递面单让你填写收件信息(EMS,填写中文即可)。接着再次排队等着进小房间。小房间里需要把护照交给工作人员+录指纹+拍照
    \item 递签当天邮箱就通知签证申请被发送去审核了
    \item 收到签证返回的邮件,快递打电话通知取签证。PS: 如果离签证中心有一定距离或者不想排队领签证的话,推荐快递服务
\end{itemize}


\subsection{衣食住行}
\href{https://hallslife.liverpool.ac.uk}{Halls Life} 是利物浦大学学生宿舍的 community hub,提供了对衣食住行的若干建议,强烈建议在出发前往英国之前浏览,可帮助决定是否携带某些行李。

\subsubsection{赴英物品(供秋冬季去的那批参考)}
\begin{itemize}
    \item 转接头*2,插排*1
    \item 小锅*1,碗*2,筷子*2
    \item 保温杯,烧水杯
    \item 冲锋衣,长款羽绒服,羊绒衫,围巾*1,帽子*1(利物浦保命必备!),防水运动鞋
    \item 塑料拖鞋(这玩意那也有,就是价格有点小贵)
    \item 贴身衣物若干,秋衣秋裤*2(真滴有必要!),睡衣*2(那边也有,质量一般呐)
    \item 小方巾*2,干发毛巾*1
    \item 折叠衣架
    \item 牙刷,牙膏
    \item 洗衣皂,洗衣袋
    \item 耳机(无线耳机必备,经常去办公室或者图书馆的可以再备个头戴式的)
    \item 眼药水
    \item 现金(50磅足矣,基本用不上)
    \item 结实的雨伞(必备!)
    \item 床单,枕套,被套,被子,枕头(床上用品也可以落地再买,由于本人没什么行李且认枕头,所以就自带了)
\end{itemize}

电话卡的选项很多。办签证的时候可以顺手领一张cmlink,但注意诈骗!!激活后就会有乱七八糟的诈骗电话,啥也别理就对了。落地英国后可以去办EE,Vodafone,O2等电话卡,有的公司会提供如免费咖啡券等福利。


\subsubsection{交通}
trip.com (携程海外版)可订全英火车、地铁、飞机、酒店。 或者这边常用的单独 app 也很多:trainline 订火车票;skyscanner 买飞机票;booking 订酒店;Uber 打车。

如果有经常坐火车的需求,可以考虑办个 railcard,每张火车票可以省 1/3 的钱。


\subsubsection{住宿}
抵达利物浦前两星期会有邮件通知,利物浦会统一安排,每批学生住宿的地点不一样,所以离学校的远近也就无法确定。但几个宿舍区离学校都不算远,步行20分钟左右能到。无法自主选定一起住的同学,耐心等待安排即可。但如果临近入住时间还没有收到邮件(发生过),可以主动联系利物浦的学生宿舍部门 \href{mailto:hallsres@liverpool.ac.uk}{hallsres@liverpool.ac.uk} 询问。


\subsubsection{办公室和ID卡}
到达之后相应的staff会告知你办公室所在位置,自行前往即可。办公室有显示器和主机。

带上护照去IT那边可以改campus code和办理学生卡。ID卡可用于进出办公室,图书馆,打印,借书。

\subsubsection{利物浦好吃的}
\begin{itemize}
    \item 张记
    \item 美味轩
    \item 老蜀人
    \item ejoy里面的煎饼
    \item 泰熙家(韩餐)
    \item turtle bay(加勒比菜)
    \item Cowshed(牛排好吃!)
    \item Bamboo(白人饭,氛围感店,适合拍照打卡)
    \item Hardware (brunch,下午茶都不错!)
    \item Johnny炸鱼薯条 (浅尝一下就行,一般,很油腻)
    \item Zizzi - Liverpool One(意大利餐)
    \item Haute Dolci Liverpool(环境好,好吃好拍)
    \item chaiiwala(一定要试试印度奶茶)
    \item sweet time bakery(想吃中式面包就买它)
    \item TSUJIRI Matcha House (抹茶党狂喜)
    \item 贡茶(每每emo就奖励自己一杯黑糖啵啵奶盖!)
\end{itemize}


\subsubsection{访问其他学校图书馆}
当你在英国四处游学访问,如果恰好在疲惫时经过一所大学的图书馆,你其实可以进去坐坐,只要你提前申请了 \href{https://access.sconul.ac.uk/sconul-access}{SCONUL Access}。

虽然不同图书馆政策不同,但大多数情况下你至少可以在成功申请后进入图书馆,有的图书馆还会提供计算机访问或图书借阅等权限。

\begin{quote}
    Q. Do I need to complete an online application for each institution I would like to visit?
    
    A. No, although you do need to indicate a library you are interested in visiting as part of your application, your approval email allows you to visit any of those libraries on the list. You do not need to reapply. \hfill 21 Oct, 2024 from \href{https://access.sconul.ac.uk/page/access-faq#Multiple%20applications}{Access FAQ | SCONUL}
\end{quote}

申请提交后会由利物浦大学图书馆处理,批准后就可以凭邮件去各大学图书馆前台办理访问卡啦~

\begin{flushright}
    2022年10月21日 by \Shiyao \\
    major update 2025年3月6日 by \Yue:补充 funding、签证、衣食住行等内容
\end{flushright} \clearpage

\section{西浦虚拟机}
\url{https://vdi.xjtlu.edu.cn/portal/webclient/index.html}
如果在学校外面,想临时快速访问学校内的资源、软件,可以使用学校的在线虚拟机。虚拟机相当于放在校内的一台电脑

\section{图书馆的文献数据库}
连接校园网后,绝大部分学校订阅了的期刊论文都可以直接在论文页面下载,但(1)有的期刊官网不能识别你的IP,甚至例如著名的Annual Review系列 (2)如果你在校外也想下论文,这时就可以用到图书馆的论文数据库

\begin{figure}[H]
    \includegraphics[width=0.7\columnwidth, center]{author-folder/Kai.Wu/library-ejournal.jpg}
\end{figure}

我用得最多的是在首页 \url{https://lib.xjtlu.edu.cn} 点击e-journal,按期刊名称搜索期刊,然后就可以跳转专有链接访问。同时,图书馆还提供了很多全文数据库,甚至还有知网等中文期刊数据库。详细的使用姿势,见图书馆往期培训 \url{https://core.xjtlu.edu.cn/course/view.php?id=905} ,或者直接本项目的GitHub里(\href{https://github.com/kaiwu-astro/xp_pgrs_unofficial_guide/tree/main/fileshare}{链接})找到PPT。

\section{苏州图书馆借书和免费送书}
想看独墅湖图书馆的书,想借苏州图书馆的书,但是太远了不想去怎么办?直接小程序借书!苏州图书馆的书可以\textbf{【免费】}送到独墅湖图书馆取书,而独墅湖图书馆的书可以直接送到学校图书馆门口。
微信搜索【苏图借书】小程序。因为医保卡就是市民卡可以直接拿去借书,如果你已经拿到了医保卡,可以直接绑定医保卡,免开卡费。然后在小程序内按提示操作即可。
% \section{学生医保}
\section{重视自己的负面情绪和抑郁倾向}
\label{sec.mental_health}

\begin{figure}[H]
    \includegraphics[width=0.6\columnwidth, center]{author-folder/Kai.Wu/mental_1.pdf}
    \includegraphics[width=0.6\columnwidth, center]{author-folder/Kai.Wu/mental_2.pdf}
    \includegraphics[width=0.6\columnwidth, center]{author-folder/Kai.Wu/mental_3.pdf}
\end{figure}

伯克利研究生会2014年发表的\href{http://ga.berkeley.edu/wellbeingreport}{研究生幸福和健康报告[链接]}显示,46\%的博士生存在心理健康问题。Nature等权威杂志发表的研究认为,
\begin{enumerate}
    \item 研究生患严重焦虑和抑郁症的比例是普通人的6倍 \href{https://www.nature.com/articles/nbt.4089}{[文献链接]}
    \item 更聪明的人更有可能患情绪障碍,如焦虑或抑郁,但不太可能寻求帮助 \href{https://www.nature.com/articles/nj7677-549a}{[文献链接]}
    \item 情绪障碍在自杀中占80-90% \href{https://www.sciencedirect.com/science/article/pii/S0160289616303324}{[文献链接]}
    \item 新冠大流行期间,每5位研究者中就有4位表现出收精神健康困扰的迹象 \href{https://www.tandfonline.com/doi/full/10.1080/21568235.2021.1992293}{[文献链接]}
\end{enumerate}

\begin{figure}[H]
    \begin{tabular}{rcl}
        \includegraphics[width=0.31\columnwidth]{author-folder/Kai.Wu/chat1of3.jpg} &
        \includegraphics[width=0.31\columnwidth]{author-folder/Kai.Wu/chat2of3.jpg} &
        \includegraphics[width=0.31\columnwidth]{author-folder/Kai.Wu/chat3of3.jpg}
    \end{tabular}
    \caption{和一个已毕业的同学聊天的记录。抑郁在我们学校博士生的发生率也比想象高得多}
\end{figure}

作为高危人群的我们,接受了大量专业知识技能的训练,却几乎从没有人教我们「如何保持精神健康」。如果你现在经常感受到紧张、焦虑、失望,甚至因此有失眠等症状,请立即正视你自己的情况,把你的健康放到第一位,同时feel free to寻求学校心理咨询室的帮助。

同时,事前预防永远比事后治疗有效。希望读到这里的同学,不管你现在是否感觉良好,都点击下面的链接,读一读Zoe Ayres在利物浦的演讲(就是上面几张PPT的出处)和她的《\textit{Managing Your Mental Health During Your PhD - A Survival Guide}》一书,健康度过研究生生涯。~~\href{https://github.com/kaiwu-astro/xp_pgrs_unofficial_guide/tree/main/fileshare}{[GitHub链接]}~~~~~\href{https://gitee.com/kaiwu-astro/xp_pgrs_unofficial_guide/tree/main/fileshare}{[不能上GitHub的用这个链接]}

\section{大家电脑上都装了什么软件?(期待你的补充)}

\KW:我来抛砖引玉说一些我一定会推荐同门师弟尝试的软件:
\begin{figure}[H]
    \includegraphics[width=\columnwidth]{author-folder/Kai.Wu/kai_mac_dock.jpg}
\end{figure}
\begin{itemize}
    \item 科研和生产力工具
    \begin{enumerate}
        \item Notion (全平台)。无比强大的新一代笔记软件。我本来是Onenote重度用户,用Onenote记了7年笔记过后偶然被介绍了Notion,在尝试了很短时间过后就决定放弃用顺手的Onenote,完全转投Notion。推荐理由:Win/Mac/Linux/网页/iOS/Android全平台支持;强大的数据库功能,可完美构建自己的GTD系统,大幅提高效率;快速的全局搜索;现代化的协作功能,甚至可以用来做团队管理和Project management;强大的分享功能,可以轻易把笔记分享成公网网页;b站有大量教程。
        % \begin{figure}[H]

            \includegraphics[width=0.8\columnwidth]{author-folder/Kai.Wu/notion.jpg}
        % \end{figure}
        \item 滴答清单 (全平台)。我用过很长时间的Wunderlist(后被微软收购变成Microsoft to-do)、iOS自带的提醒事项和其他好几种清单软件,只有滴答清单用起来最顺手、最符合直觉。搭配Notion使用,可以方便的管理好学习、生活的所有大小事。支持全平台。

        \includegraphics[width=0.8\columnwidth]{author-folder/Kai.Wu/dida.jpg}
        \item Trello (全平台)。卡片式管理你的待办事项和任务。我导师一直严重依赖Trello来把他的暴多事情管理得井井有条。虽然我认为Notion是Trello的子集,但如果你因为某些原因不喜欢Notion,至少可以尝试下Trello。全平台即可下载。

        \includegraphics[width=0.8\columnwidth]{author-folder/Kai.Wu/trello.jpg}
        \item VSCode (Visual Studio Code) (PC全平台),2019年左右兴起的开源文本编辑器,有极其强大的社区和插件支持。目前我的期刊论文、毕业论文、包括现在你看到的这一本攻略我都是用VSCode + LaTeX Workshop插件写的。我的所有数值模拟程序、数据分析脚本也都是用VSCode完成。缺点是有较小的上手难度,需要折腾配置;以及和Chrome一样占内存。

        \includegraphics[width=0.8\columnwidth]{author-folder/Kai.Wu/vscode.jpg}
        \item Spark。苹果多年的最佳App开发者Readdle做的邮件app,一站式管理所有邮箱,能把邮箱配置同步到云端,解决了每买一个新设备就要重新添加一大堆邮箱的问题。似乎是Mac和iOS独占,不确定其他平台有没有。

        \includegraphics[width=0.8\columnwidth]{author-folder/Kai.Wu/spark.jpg}
        \item 文献管理软件就不推荐了,萝卜青菜各有所爱,没有的赶紧去问师兄师姐或者上b站看推荐。
    \end{enumerate}
    \item 效率工具
    \begin{enumerate}
        \item Bob(MacOS)或Pot(Win/Mac/Linux),划词翻译、截图翻译、OCR工具。支持接入ChatGPT、DeepL、谷歌翻译、有道翻译等各种翻译接口,甚至可以做到一次调用就同时输出多个平台的翻译结果然后对比。Bob在Mac的App Store下载;Pot在pot-app.com

        \includegraphics[width=0.8\columnwidth]{author-folder/Kai.Wu/Bob.jpg}
    \end{enumerate}
\end{itemize}

你有发现什么很好用的软件吗,希望不吝赐教分享给学弟学妹

你的推荐: