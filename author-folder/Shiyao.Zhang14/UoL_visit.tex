\section{利物浦访学}
\label{section.UoL_visit}

\subsection{交流时长}
一般是3-6个月,短于3个月也行,原则上来说不允许超过6个月,如有超过六个月的,需要写明充足的理由,并获得两方导师以及 HoD 的批准。不过,利物浦那边学院的行政告诉我他们没有预算给超过六个月的部分,这个可能会是一个问题;另外,超过六个月的访学时间可能会涉及签证问题,需要提前咨询。

\subsection{关于 host supervisor}
如果导师团队中有利物浦导师,那默认就是那个老师。如果没有但又特别想去利物浦,可以自行联系利物浦那边的意向老师,要获得他/她的许可和签字。

尽管大多数同学的导师团队中会有一个来自利物浦的导师,但每个项目可能都不一样,也许你在去利物浦之前和他的互动并不频繁。这种不频繁可能是不同原因带来的,不一定是因为你工作的领域和他不同。所以如果你认为在利物浦与他合作会很有趣,那么请提前做些准备。

如果你之前和 ta 没什么互动,请务必通过某种方式建立一定程度上的熟悉感,这会使后面的合作更加顺畅。这可以是通过视频会议或邮件分享你的研究进展及计划、一起合作撰写 paper、或提前构思潜在的合作研究内容等。

我们知道团队合作有可能出于不同的原因(详见 \ref{subsection.teamwork}),但无论如何,请务必提前与你在利物浦的导师至少在大方向上确认你访问期间的活动,包括但不限于:需要他参与的事情、时间安排、需要你准备的事情、以及你准备做的事情。

\subsection{关于 funding}
西浦:学校提供1万2的经费,包含苏州往返上海办理签证的高铁票,签证费用,来回英国机票,英国当地往返利物浦的火车票(比如你落地曼城,坐火车去利物浦,就报销那一趟车票钱!以及回国时从利物浦前往机场的火车票or大巴车票),由于登机所必要的住宿。除了这些,其他的一概不报销!!而且总额不超过1.2w,超出的部分学校一概不负责。

利物浦:提供免费住宿(包bills,水电燃气随便用)+每周100磅的生活费


\subsection{签证流程}
\begin{itemize}
    \item 去官网填写相应信息
    \item 信息填写完毕后会让你选择递签地点和日期
    \item 会跳转到一个网站让你选择额外服务(比如加急或者快递),务必记住自己的申请号(GWF+一串数字)
    \item 准备相应材料,具体需要什么材料可以查看check list。所有材料均为英文。需加盖翻译公章或邀请你身边有专八证书的为你的翻译件签字 \\ \\
    护照是必备的,其他的材料包括但不限于:
    \begin{itemize}
        \item 银行流水/存款证明--我交的是存款证明,因为我在信息填写中写了存款金额,而且据说存款证明信息比较少,会加速审核流程。此处也可以递交固定资产证明。
        \item 结婚证(如果是已婚)
        \item 学校开的在读证明和资金支持(因为我是学生),已工作的人可以交工作证明
    \end{itemize}
    \item 递签--递签当日提前半小时到达签证中心即可。进去后会有工作人员引导你排队,然后进入里面将check list给工作人员,她会给你一个号码,等叫号。第一次叫号是初步询问,会问你要不要快递或短信或加急服务(短信服务可不要,因为邮箱第一时间也会有通知)。如果step 3未选择任何服务但这时要购买会给你开个单子然后让你去交钱。快递服务购买后会给你一个快递面单让你填写收件信息(EMS,填写中文即可)。接着再次排队等着进小房间。小房间里需要把护照交给工作人员+录指纹+拍照
    \item 递签当天邮箱就通知签证申请被发送去审核了
    \item 收到签证返回的邮件,快递打电话通知取签证。PS: 如果离签证中心有一定距离或者不想排队领签证的话,推荐快递服务
\end{itemize}


\subsection{衣食住行}
\href{https://hallslife.liverpool.ac.uk}{Halls Life} 是利物浦大学学生宿舍的 community hub,提供了对衣食住行的若干建议,强烈建议在出发前往英国之前浏览,可帮助决定是否携带某些行李。

\subsubsection{赴英物品(供秋冬季去的那批参考)}
\begin{itemize}
    \item 转接头*2,插排*1
    \item 小锅*1,碗*2,筷子*2
    \item 保温杯,烧水杯
    \item 冲锋衣,长款羽绒服,羊绒衫,围巾*1,帽子*1(利物浦保命必备!),防水运动鞋
    \item 塑料拖鞋(这玩意那也有,就是价格有点小贵)
    \item 贴身衣物若干,秋衣秋裤*2(真滴有必要!),睡衣*2(那边也有,质量一般呐)
    \item 小方巾*2,干发毛巾*1
    \item 折叠衣架
    \item 牙刷,牙膏
    \item 洗衣皂,洗衣袋
    \item 耳机(无线耳机必备,经常去办公室或者图书馆的可以再备个头戴式的)
    \item 眼药水
    \item 现金(50磅足矣,基本用不上)
    \item 结实的雨伞(必备!)
    \item 床单,枕套,被套,被子,枕头(床上用品也可以落地再买,由于本人没什么行李且认枕头,所以就自带了)
\end{itemize}

电话卡的选项很多。办签证的时候可以顺手领一张cmlink,但注意诈骗!!激活后就会有乱七八糟的诈骗电话,啥也别理就对了。落地英国后可以去办EE,Vodafone,O2等电话卡,有的公司会提供如免费咖啡券等福利。


\subsubsection{交通}
trip.com (携程海外版)可订全英火车、地铁、飞机、酒店。 或者这边常用的单独 app 也很多:trainline 订火车票;skyscanner 买飞机票;booking 订酒店;Uber 打车。

如果有经常坐火车的需求,可以考虑办个 railcard,每张火车票可以省 1/3 的钱。


\subsubsection{住宿}
抵达利物浦前两星期会有邮件通知,利物浦会统一安排,每批学生住宿的地点不一样,所以离学校的远近也就无法确定。但几个宿舍区离学校都不算远,步行20分钟左右能到。无法自主选定一起住的同学,耐心等待安排即可。但如果临近入住时间还没有收到邮件(发生过),可以主动联系利物浦的学生宿舍部门 \href{mailto:hallsres@liverpool.ac.uk}{hallsres@liverpool.ac.uk} 询问。


\subsubsection{办公室和ID卡}
到达之后相应的staff会告知你办公室所在位置,自行前往即可。办公室有显示器和主机。

带上护照去IT那边可以改campus code和办理学生卡。ID卡可用于进出办公室,图书馆,打印,借书。

\subsubsection{利物浦好吃的}
\begin{itemize}
    \item 张记
    \item 美味轩
    \item 老蜀人
    \item ejoy里面的煎饼
    \item 泰熙家(韩餐)
    \item turtle bay(加勒比菜)
    \item Cowshed(牛排好吃!)
    \item Bamboo(白人饭,氛围感店,适合拍照打卡)
    \item Hardware (brunch,下午茶都不错!)
    \item Johnny炸鱼薯条 (浅尝一下就行,一般,很油腻)
    \item Zizzi - Liverpool One(意大利餐)
    \item Haute Dolci Liverpool(环境好,好吃好拍)
    \item chaiiwala(一定要试试印度奶茶)
    \item sweet time bakery(想吃中式面包就买它)
    \item TSUJIRI Matcha House (抹茶党狂喜)
    \item 贡茶(每每emo就奖励自己一杯黑糖啵啵奶盖!)
\end{itemize}


\subsubsection{访问其他学校图书馆}
当你在英国四处游学访问,如果恰好在疲惫时经过一所大学的图书馆,你其实可以进去坐坐,只要你提前申请了 \href{https://access.sconul.ac.uk/sconul-access}{SCONUL Access}。

虽然不同图书馆政策不同,但大多数情况下你至少可以在成功申请后进入图书馆,有的图书馆还会提供计算机访问或图书借阅等权限。

\begin{quote}
    Q. Do I need to complete an online application for each institution I would like to visit?
    
    A. No, although you do need to indicate a library you are interested in visiting as part of your application, your approval email allows you to visit any of those libraries on the list. You do not need to reapply. \hfill 21 Oct, 2024 from \href{https://access.sconul.ac.uk/page/access-faq#Multiple%20applications}{Access FAQ | SCONUL}
\end{quote}

申请提交后会由利物浦大学图书馆处理,批准后就可以凭邮件去各大学图书馆前台办理访问卡啦~

\begin{flushright}
    2022年10月21日 by \Shiyao \\
    major update 2025年3月6日 by \Yue:补充 funding、签证、衣食住行等内容
\end{flushright}