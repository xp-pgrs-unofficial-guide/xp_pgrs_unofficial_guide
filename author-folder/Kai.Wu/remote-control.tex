\section{远程控制·如何校外访问校内电脑}

人在校外的时候,经常会很想念校内的电脑。不管是学校给大家每人配的小台式,还是用导师经费买的的其他电脑,都可以远程控制。

\subsection{学校配的电脑·准备工作}
(如果不是想访问学校配的电脑请直接跳到下一节)

学校给的电脑比较特殊,默认我们不能安装软件。有两种办法
\begin{enumerate}
    \item 发邮件或者打电话给IT,直接问他们要管理员权限。就说你是博士生,要在学校给的电脑上装软件。之后就可以装控制软件了
    \item 上面这种办法过后,电脑还是归学校管理,仍然有一些限制,但对普通使用基本够了。\sout{(有强迫症)}想完全控制学校电脑的同学\sout{(比如我)},可以直接格盘重装系统,或者分区(保留原系统)另装系统做双系统。windows重装系统、分区做双系统的教程,b站和知乎特别多。
\end{enumerate}

\subsection{桌面控制软件推荐}
以下软件不分系统,win/linux/mac均可用。
\begin{itemize}
    \item VNC: 被控端(学校电脑)安装 VNC Server \url{https://www.realvnc.com/en/connect/download/vnc/},控制端(你手上的电脑)安装 VNC Viewer \url{https://www.realvnc.com/en/connect/download/viewer/},注册账号登录使用,无需记录连接码。推荐理由:VNC是饱经检验的远程桌面,只要网络不挂,一般不会有问题。
    \item ToDesk:\url{https://www.todesk.com/}两边电脑软件一样。推荐理由:新兴软件,流畅度一般比vnc好一些。
    \item 其他:anydesk,Teamviewer,向日葵,均可尝试。其中,不太推荐Teamviewer,虽然老牌,但因为很可能检测到西浦的网络环境过后强制购买商业版,其他软件无此问题。
\end{itemize}

\subsection{远程SSH}
(不用linux的同学可直接跳到下一节看踩坑)

如果被控端是linux,你又主要使用命令行操作,那在桌面控制外,直接设法连接SSH,速度会快特别多。

电脑在学校是10开头的内网ip,如何在外面SSH?这种奇技淫巧叫做【内网穿透】

首先推荐这个视频:

\href{https://www.bilibili.com/video/BV1Qq4y1w7F5}{【硬核】公网访问?内网穿透!零经验上手!}\url{https://www.bilibili.com/video/BV1Qq4y1w7F5}

视频里校内机器可用的方案有两个,分别是

\begin{itemize}
    \item 视频的04:52,用IPV6连接。但要注意,我实测学校IPV6不稳定,过几天就会出现有v6地址但不能上v6网站的情况,更别提访问。即使你能做ddns,我个人也不推荐。
    \item 更靠谱的是视频08:09介绍的大内网穿透。免费易用的方案有:Zerotier(在境外,略慢),花生壳(国内老牌,但注册需要身份证),NOFRP(新兴,可靠性可能不高),或者直接在b站搜【免费内网穿透】,会有很多新方案。这几个方案如果不太会,b站、知乎有大量教程。
    \item (进阶大流量版,但学习成本较高)免费的方案对流量和带宽的限制都比较大,只是SSH执行命令完全足够了。但如果你想要奢侈的用scp拷文件,或者用SSH转发VNC,请考虑用导师的钱租一台云服务器,手动搭建frp服务(比较麻烦,但操作完过后用起来很爽,转发VNC比前面几种远程桌面都快),甚至,在宿舍宽带下弄个垃圾机器做跳板机,搞到公网ip,配合ddns,无限流量访问。这些方案学习成本较高,有兴趣的可以参考网上教程慢慢折腾。
\end{itemize}

\subsection{踩过的坑}
下面是我踩过大量坑过后从坑里带出来的经验:
\begin{enumerate}
    \item 冗余提高可靠性:不要只用一个远程控制方案,万一挂了一个,还能用另一个。请至少安装两个,如果是长时间不在校,可以安装更多个。我自己用的是:家宽里用一台二手瘦主机搭frp加ddns + 花生壳作backup-plan + vnc作another-backup-plan + ToDesk四重备份方案,稳定访问校内的linux电脑。
    \item 安装完成后,重启一次,看下控制软件是否能自己启动。
    \item 如果被控端是windows电脑,请务必屏蔽系统更新。虽然重启没问题,但windows大版本更新过后,会卡在一个让你同意新的用户协议的界面,除非你到场点一下,所有软件都不会启动,也就直接失去控制。请百度【禁止windows更新】。个人建议用组策略+改host组合拳。
    \item 跑程序的同学,请务必注意,不要超内存,不要炸系统内存。内存一吃满,系统立即死机,控制软件全杀完,而且还不会自己重启,必须要到场强制重启。请一定想办法在程序里监测、控制内存用量。
    \item 有条件的可以购买【向日葵控控】这样的远程KVM硬件,即使死机也可以自己远程强制重启。
    \item 再好的方案,也敌不过学校网络维护断网和学校施工断电。而这些,也是可以防御的。应对断网,请注意在网络登录时勾选macauth,网络维护完了就又能用了。来电自动启动功能,需要主板支持,在主板说明书搜索boot on power或boot with power,或直接搜power,看看有没有。或者,购买向日葵控控。不过断电断网始终是极少数时候,除非你一走几个月,基本不需要考虑\sout{(但万一呢)}。
    \item 请和至少一个在校同学搞好关系,新手总会遇到各种意外的情况失去控制,需要办公室同学帮你跑腿看机器情况。100种远程控制的方案背后,也需要好同学做物理控制。万一的万一,哪天网线被人不小心踢掉了呢,还是只有\sout{通过py交易}让同学帮你弄。
\end{enumerate}

\begin{flushright}
(2022年11月09日 by \Wu)
\end{flushright}

% \begin{figure}[H]
%     \centering
%     \includegraphics[width=0.5\columnwidth]{author-folder/Kai.Wu/}
% \end{figure}


% \usepackage[export]{adjustbox}

% \item 
% \begin{minipage}{0.3\textwidth}
%     文字
% \end{minipage}
% \begin{minipage}{0.63\textwidth}
%     \begin{figure}[H]
%         \includegraphics[width=0.95\columnwidth, right]{author-folder/Kai.Wu/}
%     \end{figure}
% \end{minipage}

% \input{author-folder/Kai.Wu/.tex}
