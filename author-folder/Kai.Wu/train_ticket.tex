\section{火车票·学生票优惠}

和本科生一样,我们也是能享受学生票优惠的。步骤如下

\begin{enumerate}
    \item 申请学生证。学生证,不是你的ID卡,是红色的小本本。博士生,默认是不发学生证的,要自己申请才发。方法:在一站式网站 \url{https://studentonestop.xjtlu.edu.cn/} 里的申请/补办学生证,按提示操作。学生证不仅可以用于买火车的学生票,也可用于各景点买学生票(例如拙政园,买学生票过后进门查学生证,ID卡是不行的)。
    \item 拿到的学生证,最后一页会有一个火车票优惠卡。领取学生证的时候,跟一站式老师double check一下这个优惠卡可以直接用了,就OK了(主要是确认他们给你这个优惠卡在学校层面激活了)。
    \item 后面要注意,在火车上如果遇到列车员double check你学生身份,发现你学生证注册章不是最新的,按政策是会要求你补全票的。所以,坐火车之前,要注意注册章要盖够了。也不必每学期都去盖,除非每学期都要回家。我的做法是,买学生票的时候再去check自己注册章有没有盖够,否则立马去一站式补就行(可以一口气盖一堆)。
    \item 
        \begin{minipage}{0.71\textwidth}
            如何使用优惠卡?如何购买学生票?有什么限制?因为政策可能变动,请直接关注官方公众号,在公众号菜单里有常见问题FAQ。
        \end{minipage}
        \begin{minipage}{0.2\textwidth}
            \begin{figure}[H]
                \includegraphics[width=0.95\columnwidth, center]{author-folder/Kai.Wu/qrcode_huitongstudent_1.jpg}
            \end{figure}
        \end{minipage}

\end{enumerate}


\begin{flushright}
(2023年01月04日 by Kai Wu)
\end{flushright}

% \begin{figure}[H]
%     \centering
%     \includegraphics[width=0.5\columnwidth]{author-folder/Kai.Wu/}
% \end{figure}


% \usepackage[export]{adjustbox}

% \item 

% \begin{newminipage}[0.65]
%     文字
% \end{newminipage}
% \begin{newminipage}[0.34]
%     \begin{figure}[H]
%         % \caption{}
%         \includegraphics[width=0.95\columnwidth, right]{author-folder/Kai.Wu/}
%     \end{figure}
% \end{newminipage}

% \input{author-folder/Kai.Wu/.tex}
