\section{助教(Teaching Assistant)怎么做}

\subsection{我是否需要做TA?}
\begin{itemize}
    \item 如果你是有奖学金的博士生,那根据学校发给你的的Financial offer,你每学期都有TA的义务(换句话说,奖学金是TA工作量换来的,拿了钱必须做事)。做了TA过后,不会另外结算工资。
    \item 如果你是自费生,那你完全可以不做TA。但做TA过后,学校会按照工时给你发工资,而且TA的经历写到你简历上,(视你未来的职业规划)可能有一些作用。
\end{itemize}

\subsection{如何参加TA}
每学期的第一周前后,学院会招募TA,一般由学院的秘书(School Academic Administrator)联系大家。如果你很想参加,或者想早点占某个科目的坑,可以尽早联系秘书以及对应的科任老师。

\subsection{TA的工作种类,以及分别怎么做}

一般来说,TA工作会对应下面的一项或多项工作

\subsubsection{批作业}
如果是传统的改线下的纸质作业,一般问老师怎么做就行。如果是通过Learning Mall(以下简称LMO)提交的电子版作业,下面是教程:
\begin{enumerate}
    \item 登录LMO,找到你做助教的课程。如果翻遍了都找不到,需要发邮件给你们学院的秘书或者任课老师给你权限。
        \begin{figure}[H]
            \centering
            \includegraphics[width=0.5\columnwidth]{author-folder/Kai.Wu/LMO_course.png}
        \end{figure}
    \item 点进课程,往下拉找到提交作业的地方,或者直接用Ctrl+F(Mac:Command+F)搜索 submission
        \begin{figure}[H]
            \centering
            \includegraphics[width=0.4\columnwidth]{author-folder/Kai.Wu/LMO_submission_links.png}
        \end{figure}
    \item 超大的必修课会有很多个班,先选择你被分配到改作业的班级(Seperate group)。接下来,如果你要马上开始在网页上在线批改,点击Grade。如果想看看,或者想离线批改(比如下载到iPad上),点击view all submission。
        \begin{figure}[H]
            \centering
            \includegraphics[width=0.5\columnwidth]{author-folder/Kai.Wu/LMO_inside_submission.png}
        \end{figure}
    \item 我对在线批改系统深恶痛绝,很卡,而且在线的PDF标记工具贼难用,改作业效率很低。除非老师做了打分表格(我导师给我展示过一次,在线批改的时候右边可以直接选择每个小题的分数,以及错误原因,这样就很高效),但我改过作业的module一个都没有做过这个,所以在线就很鸡肋。因此下面我只介绍我摸索出来的相对高效的离线批改方法。点进view all submission过后,在上面的grade action里,分别选择 下载成绩表 和 下载全部提交文件
        \begin{figure}[H]
            \centering
            \includegraphics[width=0.5\columnwidth]{author-folder/Kai.Wu/LMO_download.png}
        \end{figure}
    \item 然后你就会获得一个csv文件和一个超级大的zip。解压zip会看到所有学生提交的作业都以姓名+学号+一堆字命名好了。接下来你就可以把这堆文件在你电脑上、或者传到平板上本地批改。如果你的科任老师不要求把批改过的作业作为feedback file发回给学生(得问老师),你甚至可以全部打印出来改。改完过后,把成绩登到成绩表里。csv可以用Excel打开,你可以另存为xlsx格式。表格最后一列是Feedback comments,可以在里面写对学生要说的话,比如,文件格式不对、下次请拍清晰一点,之类的
        \begin{figure}[H]
            \centering
            \includegraphics[width=0.5\columnwidth]{author-folder/Kai.Wu/LMO_Downloaded.png}
        \end{figure}
    \item 按照老师的要求改完作业过后,可以很方便的把成绩表和批改过的文件(如果老师要求)传回LMO。还是在刚才view all submission之后的页面,点击upload grading worksheet,把成绩表传上去。
        \begin{figure}[H]
            \centering
            \includegraphics[width=0.5\columnwidth]{author-folder/Kai.Wu/LMO_upload.png}
        \end{figure}
    \item 要注意,如果你中间用Excel把它存成了xlsx,上传前需要另存为csv格式再上传。勾选下面allow那一句,允许表格内容覆盖已经在网页改过的内容。点击下面upload过后,你会看到很长一串网页,每个学生的成绩(和你输入的feedback comments)就传上去了。
        \begin{figure}[H]
            \centering
            \includegraphics[width=0.5\columnwidth]{author-folder/Kai.Wu/LMO_upload_sheet.png}
        \end{figure}
    \item 回到刚才的页面,点upload multiple feedback files in a zip,上传批改过的作业文件。把作业文件打包成zip上传,但由于LMO限制上传大小不能超过100M,如果总大小超过了就要手动打包成多个小于100M的zip(神烦,为什么下载就可以超过100M)。这里提供一个我随便写的python脚本 \href{https://github.com/kaiwu-astro/xp_pgrs_unofficial_guide/tree/main/fileshare/zip_in_100M.py}{GitHub链接}或\href{https://cowtransfer.com/s/8a91e9aa59ad4a}{不能上GitHub的用这个链接},来自动按最大100M打包成多个文件,mac和linux可用,win下要注意路径名(如果不会python当然还是手动打包比较快)。上传完过后就完事了。
    \item 最后要发邮件给老师,说明(1)发现的共性问题,比如某道题错得很多,某道题不会的很多,某道题有的学生错理解成了什么,之类的,(2)个别问题,例如某学生交错了文件,某学生疑似抄袭标准答案(这种一般是高年级学生给他的),某学生和某学生作业雷同,(3)以及其他任何想和老师沟通的问题。大学数学系有下面这种专门的反馈表,填到表里。如果没有反馈表的,给老师发邮件说下也可以。(一般都不用太详细,不要学我,很费时间,除非你热爱这项事业)
        \begin{figure}[H]
            \centering
            \includegraphics[width=0.5\columnwidth]{author-folder/Kai.Wu/LMO_feedback_to_teacher.png}
        \end{figure}
\end{enumerate}


\emptyline
【分享一下我的批作业经验】
\begin{enumerate}
    \item 我的工具组合是iPad + Apple pencil + PDF Expert app,比在电脑上在线、离线改都快得多。另外如果购买磁吸类纸膜(约20元)+pencil替换笔头(约10元),会大幅提高书写体验
        \begin{figure}[H]
            \centering
            \includegraphics[width=0.5\columnwidth]{author-folder/Kai.Wu/marking_tools.jpg}
        \end{figure}
    \item 流水线提高效率:题多的时候,不要全部改完一个学生的作业再改下一份,这样尤其慢。尝试一下把作业分成几部分,先把所有作业的part 1(比如,前面三道题)改完,然后回过头来,改所有人的部分part 2(比如4-6题)。可能打开关闭文件会费一些时间,但这样能记住几个题的答案和评分点,改起来很快。切忌改每个作业都去看几眼答案。
    \item 同理,不要改完一个就登一个成绩。我的习惯是,把总成绩写一个巨大数字在左上角。全部改完过后,传电脑上,用最大号的缩略图(Mac:显示为图标,狂按command+等号;win:显示为超大图标),直接不打开文件就能看到左上角的分数。Excel里按姓名排序,文件夹里用名称排序,统起成绩来特别快。
        \begin{figure}[H]
            \centering
            \includegraphics[width=0.7\columnwidth]{author-folder/Kai.Wu/tongchengji.png}
        \end{figure}
    \item 学生提交的文件可能有很多大病。比如,提交的不是pdf而是word甚至ppt、提交的pdf里面图片顺序是乱的、图片方向是错的、有标准的习题册不用而是自己拿个本写导致不好对答案。TA有权要求学生交作业的格式,可以对这些学生在feedback comments里面警告,并且提醒科任老师一定要在课上提醒学生,否则真的非常影响批阅效率。学生警告了仍然乱交格式时,可以按你的标准扣格式分。比如,我是初犯警告,再犯扣20\%,三次或以上扣50\%。
    \item 拿不准给多少分合适的时候,就给多一点。给少了,学生会找你argue,很浪费时间。给多一点,相当于节约自己的时间。
    \item 学生有可能会通过LMO给你发消息,或者给你发邮件。尽量不要直接回复学生,毕竟回复错了的话老师背锅。任何学生发给你的消息,请转给老师,让老师回复他。除非老师同意你直接回复。
\end{enumerate}

\subsubsection{批改试卷}
Quiz,期中,期末考试,助教有可能要参加评卷。所以期末放假跑路之前跟老师问清楚,以免要改期末卷但是不知道。如果确实有事要提前跑路,和老师商量。
\subsubsection{讲Tutorial(习题课)}
(我没带过习题课,谁讲过的来介绍下经验)
\subsubsection{带Lab(实验课)}
(挖坑待补)

\emptyline{}
另外,期中、期末考试的监考严格来说也是TA的一种,会计入奖学金学生的TA工时,对自费同学,监考的工资一般也丰厚得多。监考见下一节。


\begin{flushright}
(2022年10月21日 by Kai Wu)
\end{flushright}

% \begin{figure}[H]
%     \centering
%     \includegraphics[width=0.5\columnwidth]{author-folder/Kai.Wu/}
% \end{figure}

% \begin{wrapfigure}{r}{0.25\textwidth} 
%     \centering
%     \includegraphics[width=0.25\textwidth]{mesh}
% \end{wrapfigure}